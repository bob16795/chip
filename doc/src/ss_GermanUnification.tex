%-\documentclass[12pt]{article}
%-
%-\usepackage[margin=.50in]{geometry}
%-\usepackage[backend=biber, style=authoryear-icomp]{biblatex}
%-\usepackage{ot-tableau}
%-\usepackage{easylist}
%-\usepackage{hanging}
%-\usepackage{hyperref}
%-\usepackage{blindtext}
%-\usepackage{tipa}
%-\usepackage{cgloss4e}
%-\usepackage{gb4e}
%-\usepackage{qtree}
%-\usepackage{enumerate}
%-\usepackage{longtable}
%-\addbibresource{$HOME/doc/uni.bib}
%-
%-\title{German Unification}
%-\author{Preston Precourt}
%-
%-\begin{document}
%-
%-	\maketitle

	\begin{enumerate}
		\item people used nationalism as a mean of justificatio n for inhumane actions.
		\item separation.
		\item two diferent stances on govt.
		\item they lacked motivation.
		\item Separation.
		\item lacked motivation.
		\item germany.
		\item they were nationalistic.
		\item he thought if people used more energy on soething they would think its better.
		\item no so then they could conqure them
		\item no.
	\end{enumerate}

%	\printbibliography

%-\end{document}
