%\documentclass[12pt]{article}
%
%\usepackage[margin=.50in]{geometry}
%\usepackage[backend=biber, style=authoryear-icomp]{biblatex}
%\usepackage{ot-tableau}
%\usepackage{easylist}
%\usepackage{hanging}
%\usepackage{hyperref}
%\usepackage{blindtext}
%\usepackage{tipa}
%\usepackage{cgloss4e}
%\usepackage{gb4e}
%\usepackage{qtree}
%\usepackage{enumerate}
%\usepackage{longtable}
%\addbibresource{$HOME/doc/uni.bib}
%
%\title{Italian Unification}
%\author{Preston Precourt}
%
%\begin{document}
%
%\maketitle

\begin{enumerate}
  \item The Austrian and french forces worked together to crush the romantic Italian nationalist's.
  \item Mazzini formed Risorgimento and Young Italy.
  \item He opposed Austrian control.
  \item He broke a truce that left Venetia in Austria's grip.
  \item They secured the independence of Italy.
  \item He was worried it would be abused. He gave it to King Victor.
  \item The pope refused to recognize Italy.
  \item Economic differences.
\end{enumerate}

\begin{tabular}{| l  | p{10cm}|}
	\hline
	Italian Leader & Achievements and significance \\ \hline \hline
	Giuseppe Mazzini & He formed Risorgimento and Young Italy and tried to
	rid Itialy from Austrian Influence\\ \hline
	Camillo & He opposed Austrian control \\ \hline
	Guissupeoieepeoeio & secured the independence of Italy. \\ \hline
	Victor & The first king of itialy \\ \hline
\end{tabular}

%\printbibliography

%\end{document}
