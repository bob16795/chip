\documentclass[avery5388,grid,frame]{flashcards}
\cardfrontstyle[\large\slshape]{headings}
\cardbackstyle{empty}

\begin{document}

\cardfrontfoot{Science}

\begin{flashcard}[Definition]{Neutron}
One of the 3 smallest particles
  \smallskip
  \begin{description}
    \item[fact] The neutron has no a mass of 1 amu
    \item[fact] The neutron has a polarity of zero 
    \item[fact] The neutron resultsd in stuff 
  \end{description}
\end{flashcard}

\begin{flashcard}[Definition]{Electron}
One of the 3 smallest particles
  \smallskip
  \begin{description}
    \item[fact] The Protron has no a mass of 0 amu
    \item[fact] The Protron has a polarity of -1 
    \item[fact] The Protron resultsd in stuff 
  \end{description}
\end{flashcard}

\begin{flashcard}[Definition]{Protron}
One of the 3 smallest particles
  \smallskip
  \begin{description}
    \item[fact] The Protron has no a mass of 1 amu
    \item[fact] The Protron has a polarity of +1 
    \item[fact] The Protron resultsd in stuff 
  \end{description}
\end{flashcard}

\begin{flashcard}[Definition]{Atom}
  the smallest defined state of matter
  \smallskip
  \begin{description}
    \item[fact] Has a Polarity of zero
    \item[fact] You cant see them
  \end{description}
\end{flashcard}

\begin{flashcard}[Definition]{Ion}
  A charged atom
  \smallskip
  \begin{description}
    \item[fact] Has a polarity of something othe than zero
    \item[fact] You cant see them
  \end{description}
\end{flashcard}

\begin{flashcard}[Definition]{Isotope}
  Two atoms with diferent amnt of Neutrons but the same ammount of Protorons
  \smallskip
  \begin{description}
    \item[fact] every element has an isotope
  \end{description}
\end{flashcard}

\begin{flashcard}[Definition]{Atomic Number}
  The Defining property of an element
  \smallskip
  \begin{description}
    \item[fact] Every element has its own atomic number
  \end{description}
\end{flashcard}

\begin{flashcard}[Definition]{Chemical Change}
  A change in witch a product is not an output
  \begin{description}
    \item[fact] Burning is not an example 
  \end{description}
\end{flashcard}

\begin{flashcard}[Definition]{Phycial change}
  A change in witch an element doesnot change.
  \begin{description}
    \item[example] State of mater
  \end{description}
\end{flashcard}

\begin{flashcard}[Definition]{Allotrope}
  Two molecules with the same formula
  \begin{description}
    \item[fact] no facts
  \end{description}
\end{flashcard}

\begin{flashcard}[Definition]{ionization energy}
  the energy required to produce the ion
  \begin{description}
    \item[fact] ions are cool
  \end{description}
\end{flashcard}

\begin{flashcard}[Definition]{Electronegitivity}
  Desire for electrons
  \begin{description}
    \item[fact] Hidrogen has low Electronegitivity
  \end{description}
\end{flashcard}

\begin{flashcard}[Definition]{Compound}
  A mixtur of two ore more elements
  \begin{description}
    \item[fact] Coffie is a mixture
  \end{description}
\end{flashcard}

\begin{flashcard}[Definition]{Empirecial formula}
  A formula that is fully simpliffied.
  \begin{description}
    \item[fact] none
  \end{description}
\end{flashcard}

\begin{flashcard}[Definition]{Ionic bond}
  a bond between tow atoms
  \begin{description}
    \item[fact] elecrtons are transfered 
  \end{description}
\end{flashcard}

\begin{flashcard}[Definition]{Covelant bond}
  a bond between a non metal and a metal
  \begin{description}
    \item[fact] H2O is one
  \end{description}
\end{flashcard}

\begin{flashcard}[Definition]{AKE}
  Average kenetic energy is Temprature
  \begin{description}
    \item[fact] Take
  \end{description}
\end{flashcard}

\begin{flashcard}[Definition]{Excited state}
  A state of an ion in witch a elecrton is moved
  \begin{description}
    \item[fact] There are infinite Excited states
  \end{description}
\end{flashcard}

\begin{flashcard}[Example]{Single Replacement reaction}
  AB + C -> AC + B
\end{flashcard}

\begin{flashcard}[Definition]{Double Replacement reaction}
  AB + CD -> AC + BD
  \begin{description}
    \item[fact] lololololol
  \end{description}
\end{flashcard}

\begin{flashcard}[Definition]{Decomposition reaction}
  AB -> A + B
  \begin{description}
    \item[fact] lnope
  \end{description}
\end{flashcard}

\begin{flashcard}[Definition]{Synthsys reaction}
  A + B -> AB
  \begin{description}
    \item[fact] none
  \end{description}
\end{flashcard}

\begin{flashcard}[Definition]{Atomic Mass}
  The mass of an atom in AMU
  \begin{description}
    \item[fact] Diferent isotopes have diferent atomic masses
  \end{description}
\end{flashcard}

\begin{flashcard}[Definition]{Hydrogen Bonding}
  The strongest bond
  \begin{description}
    \item[fact] H2O FH and NH3
  \end{description}
\end{flashcard}

\begin{flashcard}[Definition]{Metal}
  An element on the left of the periodic table
  \begin{description}
    \item[fact] They have less electrons
  \end{description}
\end{flashcard}

\begin{flashcard}[Definition]{Nonmetal}
  An element on the right of the table
  \begin{description}
    \item[fact] Has alot of electrons
  \end{description}
\end{flashcard}

\begin{flashcard}[Definition]{Metaloid}
  in between metals and nonmetals
\end{flashcard}

\begin{flashcard}[Definition]{Potential energy}
  Its Obvious
\end{flashcard}

\begin{flashcard}[Definition]{Heat Flow}
  Heat flows to cold
  \begin{description}
    \item[fact] When your hot your cold
  \end{description}
\end{flashcard}

\begin{flashcard}[Definition]{Hetrogenous}
  An mixture of elements that is not well mixed
\end{flashcard}

\begin{flashcard}[Definition]{Homogenoius}
  Ano mixture of elements thats well mixed
\end{flashcard}

\begin{flashcard}[Definition]{Nonplar}
  Non ploar molecules are asymetrical
  \begin{description}
    \item[fact] 
  \end{description}
\end{flashcard}
\end{document}
